% ---------------------------------------------------------------------
% -------------- PREAMBLE ---------------------------------------------
% ---------------------------------------------------------------------
\documentclass[12pt,a4paper,finnish,oneside]{article}
\usepackage[utf8]{inputenc}   % merkistökoodaus, jos käytetään UTF8:a
\usepackage{ae,aecompl}       % ed. lis. vektorigrafiikkana bittikartan sijasta
\usepackage[english,finnish,swedish]{babel}
% Kurssin omat asetukset aaltosci_t.sty:
\usepackage{aaltosci_t}

\usepackage{alltt}
\usepackage{amsmath}   % matematiikkaa
\usepackage{calc}      % käytetään laskurien (counter) yhteydessä (tiedot.tex)
\usepackage{eurosym}   % eurosymboli: \euro{}
\usepackage{url}       % \url{...}
\usepackage{listings}  % koodilistausten lisääminen
\usepackage{algorithm} % algoritmien lisääminen kelluvina
\usepackage{algorithmic} % algoritmilistaus
\usepackage{hyphenat}  % tavutuksen viilaamiseen liittyvä (hyphenpenalty,...)
\usepackage{supertabular,array}  % useampisivuinen taulukko
\usepackage[font=footnotesize,labelfont=bf]{caption}
\usepackage[iso,german]{isodate}
\usepackage[usenames,dvipsnames]{xcolor}
\usepackage[color=YellowGreen]{todonotes}
\usepackage{listings}

\lstset{basicstyle=\footnotesize\ttfamily}
\lstset{captionpos=b}

% Rangaistaan tavutusta (ei toimi?! Onko hyphenat-paketti asennettu?)
\hyphenpenalty=10000   % rangaistaan tavutuksesta, 10000=ääretön
\tolerance=1000        % siedetään välejä riveillä

\bibpunct{[}{]}{;}{n}{,}{,}    % n = numero [1],[2] (numerical style)

% Rivivälin muuttaminen:
\linespread{1.24}\selectfont               % riviväli 1.5
%\linespread{1.24}\selectfont               % riviväli 1, kun kommentoit pois

% ---------------------------------------------------------------------
% -------------- DOCUMENT ---------------------------------------------
% ---------------------------------------------------------------------

\begin{document}

% -------------- Tähän dokumenttiin liittyviä valintoja  --------------

\input{makroja}       % Haetaan joitakin makroja

\selectlanguage{english}

\pagestyle{plain}
\pagenumbering{arabic}

\author{Oskari Virtanen}

% Otsikko nimiölehdelle. Yleensä sama kuin seuraavana oleva \TITLE, 
% mutta jos nimiölehdellä tarvetta "kaksiosaiselle" kaksiriviselle
\title{Architecture of distributed systems and CouchDB case study}
% 2-osainen otsikko:
%\title{\LaTeX{}-pohja kandidaatintyölle \\[5mm] Pitkiä rivejä kokeilun vuoksi.}

% Ohjaajan laitos suomi/ruotsi ja tarvittaessa eng (tiivistelmän kieli/kielet)
\DATE{\today}
\maketitle             % tehdään nimiölehti

% -------------- Tiivistelmä / abstract -------------------------------
% Lisää abstrakti kandikielellä (ja halutessasi lisäksi englanniksi).

% Edelleen sivunumerointiin. Eräs ohje käskee aloittaa sivunumeroiden
% laskemisen nimiösivulta kuitenkin niin, että sille ei numeroa merkitä
% (Kauranen, luku 5.2.2). Näin ollen ensimmäisen tiivistelmän sivunumero
% on 2. \maketitle komento jotenkin kadottaa sivunumeronsa.
\setcounter{page}{2}    % sivunumeroksi tulee 2

\newpage                       % pakota sivunvaihto

% -------------- Sisällysluettelo / TOC -------------------------------

\tableofcontents

\clearpage                     % kappale loppuu, loput kelluvat tänne, sivunv.
%\newpage

% -------------- Symboli- ja lyhenneluettelo -------------------------
% Lyhenteet, termit ja symbolit.
% Suositus: Käytä vasta kun paljon symboleja tai lyhenteitä.
%

\section{Introduction}

CouchDB\footnote{http://couchdb.apache.org/} is a distributed database that is
designed around web technologies. Its API relies completely on HTTP protocol and
documents are stored in JSON format. CouchDB also promises to be highly
available, partition tolerant and eventually consistent.

Distributed systems are often categorized to two categories based on their
attributes. The categorization is made using CAP
theorem\cite{gilbert2002brewer}. CAP theorem states that it is impossible for a
web service to simultaneously guarantee consistency, availability and partition
tolerance. These properties are highly desirable for a distributed system.
Consistency in the CAP theorem means that there exists an order of operations
that is equivalent as if all the operations were completed by a single
instance. In practice this means that the results of operations the distributed
system completes need to be communicated to the every node of the system. For
CAP theorem to consider a distributed system available, it needs to response to
every request received by non-failing node. Partition tolerance means that the
system needs to tolerate partitions in the network i.e.\ the network
can drop arbitrary amount of packages and the system should still remain
functional. 

Since perfect network is impossible to build, network partitions will occur.
Based on the CAP theorem, a distributed system can not be consistent and
available at the same time, so distributed systems need to choose whether they
keep their data consistent at all times and refuse to response to some of the
queries, or remain available during partition but give up on the data
consistency.


\section{Architecture of distributed systems}

This section introduces and analyzes common design patterns of distributed
systems. It is divided into two subsections based on
CAP-theorem~\cite{gilbert2002brewer} that states that no distributed system can
achieve consistency, availability and partition tolerance at the same time. Due
the nature of distributed systems network partitions are bound to happen, thus
designers of distributed system need to choose whether the system remains
available during the partition or keeps the data consistent.

\subsection{CP-systems}

\begin{figure}[h!]
  \centering
    \includegraphics{pictures/cp_system.png}
  \caption{CP-system with a single master node and two slaves}
\label{cpimage}
\end{figure}

CP-systems endorse data consistency over availability. In practice this means
that when a network between two or more database nodes malfunctions, the
system should stop handling both, read and write requests from clients and wait
until the network is healed. Failure to do so might bring the systems into an
inconsistent state, where nodes of the system have different view of the same
datum. Consider figure~\ref{cpimage} that shows a CP-system consisting of a
single master node \(A\) and two slaves \(B\) and \(C\). Clients are able to
read from all of the nodes, but only node \(A\) accepts writes. After a
successful write, \(A\) replicates the write to the slave nodes. If the
connection between \(A\) and \(B\) fails, the system is partitioned into two
sets, one containing \(A\) and \(C\), and other containing only \(B\). Now if
the client first writes a new document to \(A\) and then tries to read it from
both \(B\) and \(C\), what happens is that \(C\) correctly returns the written
document, but \(B\) reports that the document can not be found. This happens
because master \(A\) was not able to replicate the new write to node \(B\)
because the network between the nodes was down. Because now \(B\) and \(C\) have
a different view of the same datum, the system can not be considered as
CP-system anymore. The situation can be fixed by not accepting a write from the
client, when the master node detects that it is not able to replicate the writes
to all nodes. 

As in the example, a common way to model CP-systems is a master/slave
architecture, where one node at a time acts as a master and other nodes are
slaves. To achieve consistency, only master is able to accept write requests and
depending on the architecture, slaves might be able to accept read requests. If
slaves handle the read requests, the distributed system might not be consistent
at all times. For example, node \(A\) accepted a write but did not had time to
replicate it to node \(B\) when it replies to the read request accessing that
object, the database would incorrectly report that object is not found. However
after \(A\) has replicated the object to \(B\) the database is again in
consistent state. This is called eventual consistency.

Vogels specifies eventual consistency as a specific form of weak
consistency~\cite{vogels2009eventually}. A system with weak consistency does not
guarantee that subsequent accesses to the updated object always returns the
updated object but a set of conditions need to be met before. The period between
when the object is updated and when it is guaranteed is called
\emph{inconsistency window}.

In eventual consistency, the system guarantees that if no new updates are made
to the object, eventually all access will return the updated object. In this
case, the maximum inconsistency window in best case scenario can be calculated
from different factors of the system, such as communication delay between nodes
and the load of the system. Of course in exceptional events such as network
partition the inconsistency window can grow as large as the event occur.

The problem that many CP-systems face is what happens when the master node
becomes unavailable. That may occur for many reasons, for example the node can
suffer from hardware failure or the network between master and other nodes goes
down. Naturally the system needs to be able to recover from such cases and many
different procedures have been developed to cope with the problem. Next we
discuss how Raft consensus algorithm handles election of a new leader.

Raft is a consensus algorithm developed to replace more complex algorithm Paxos.
Raft aims to be more understandable and better uncoupled than its
predecessor~\cite{ongaro2014search}. Consensus algorithms allow collection of
services to agree on some state even when some of them fails. They often arise
in context of \emph{replicated state machines}, where each service compute
identical copies of the same data. To achieve consistency, Raft selects a
distinguished leader which has a complete responsibility of managing the
replicated log and replicating them to other services. When the leader fails or
becomes unavailable, the new leader must be elected.

Raft detects leader failures with heartbeats. Each node starts as a follower and
remains in that state as long as it receives heartbeat signals from the leader.
If a follower receives no heartbeat in a period of time called \emph{election
timeout}, it assumes there are no leader and begins the leader election.

In the beginning of leader election, the follower that received no heartbeat
promotes itself to a candidate and sends vote request messages to the other
clients. A client grants the vote for the first candidate requesting the vote
and declines the rest. The candidate continues in this state until one of the
three things things happens:

\begin{enumerate}
  \item The candidate wins the election and promotes itself to the leader if it
  receives votes from majority of the cluster. Once the candidate wins the
  election it starts sending heartbeats to the other nodes and establishes the
  authority and prevents new elections.
  \item While waiting for the votes, the candidate might receive a heartbeat
  from other node. In this case it demotes itself to the follower state and
  recognizes the new leader.
  \item Last outcome is that the candidate does not win or lose the elections.
  If many followers become candidates at the same time, it is possible that none
  of the candidates receive majority of votes and initiate a new round of
  vote requests. Raft uses randomized election time outs to ensure that no two
  clients are candidates indefinitely. Another candidate might still be waiting
  out the time out when other candidate receives a majority of the votes and
  starts sending heartbeats and ending elections.
\end{enumerate}

The actual leader election in RAFT is a bit more complicated than described
above, having a concept of \emph{terms} to measure time as a logical clocks, but
they were left out for clarity. The original paper explains the election in more
detail and the website of Raft
algorithm\footnote{\url{https://raftconsensus.github.io/}} has an excellent
interactive visualization about the leader election.

\subsection{AP-systems}

\begin{figure}[h!]
  \centering
    \includegraphics[scale=0.7]{pictures/ap_system.png}
  \caption{AP-system where client 1.\ sends a write request which is replicated
  to all the nodes and is read afterwards by client 2.}
\label{apimage}
\end{figure}

AP-systems choose to be available in the event of network partition instead of
keeping the data consistent at all times. In practice this means that all nodes
of the cluster are allowed to handle both read and writes requests from clients
and there is no dedicated master node. After receiving the request, the
node can either handle the request itself or delegate it to more appropriate
node for processing. Depending from the architecture, the write may be
acknowledged when the handling node has received it, or it might require
acknowledges from multiple nodes. After a successful write of an object, the
object is replicated to other nodes for a better durability and availability.
Even if a node that originally handled the write request is unavailable, the
cluster is able to respond to the request.

Figure~\ref{apimage} shows an AP-system with four nodes and two clients. First
client \(1\) sends a write request to the node \(B\) which acknowledges it
immediately, displayed as a dashed line. It then replicates the object to the
other nodes. Later client \(2\) sends a read request for the same object that
can be handled by \(D\) since it has received the object. Next consider a case
where the network is split into two different groups, client \(1\) and nodes
\(A\) and \(B\) belong to the first group and client \(2\) and nodes \(C\) and
\(D\) to second.

Because AP-systems promote availability over consistency, the system should be
able to accept writes even if the network in the cluster is malfunctioning. In
the situation of the example but with a partitioned network, client \(1\) had
sent the write request, and the write would have been acknowledged normally.
However \(B\) would have been unable to replicate the object to the nodes in the
other side of the partition and client \(2\) had got \texttt{not\_found}
response from node \(D\). This shows an important characteristics of AP-systems,
because consistency is sacrificed to obtain better availability, the service
might get into an inconsistent state but still return successful response codes
to the user. In contrast, CP-system would have rejected the write request from
\(1\) because it could not reach other nodes thus making the system unavailable.

Another interesting problem arises when a network partition heals. Assume that
both sides of the partition have received an update to the same object. Because
AP-systems do not require the write is acknowledged by majority of nodes, both
sides might end up with its own version of the same object. When the network
heals, both sides try to replicate their version to the other side but run into
a problem. Which version is correct?

Some systems solve the problem by tracking the time when an update was received
and then choosing an object with higher timestamp. This solves the problem, but
effectively destroys updates received by the other object. The system might keep
the discarded version safe so the user can try to manually merge the two
versions later. It might be tempting to try to write an universal merge function
that merges the changes from both objects in all cases. However it can be shown
that such function does not exists\todo{how is this done?}.

Because it is impossible for the AP-system to recover from all conflict cases, a
careful thought should be placed on the conflict detection. Although concurrent
updates by different clients always end up in conflict, it is possible to
automatically solve serializable updates if detected correctly. A naive way to
detect a conflict between two versions is to compare the hashes of the versions
and check whether they differ. However this approach might lead into a false
positives that can be eliminated.

Consider a case where two clients are connected to the cluster containing two
nodes. Each time a cluster receives a write request, it immediately acknowledges
it to the client and then goes to replicate it to other clients. Each node is
also allowed to handle read requests.

\begin{figure}[h!]
  \centering
    \includegraphics[scale=0.7]{pictures/apexample.png}
  \caption{Two clients updating the same object during network partition}
\label{ap-example}
\end{figure}

Figure~\ref{ap-example} presents two clients are updating the same object when a
network partition occurs.  Before the partition, client \(1\) first writes the
object which is handled by node \(A\). Node \(A\) is also able to replicate the
write to node \(B\) before the partition occurs. During the partition, client
\(1\) updates the object which handled again by node \(A\) but this time it is
not able to replicate the write to the another node. Next client \(2\) reads the
object from \(A\) and then sends an update request for that object which is
handled by \(B\). Now when the network partition heals, nodes \(A\) and \(B\)
have different versions of the same object and mark it as conflicted.

This is correct behaviour, but as we can see from the image, the conflict could
have been solved automatically. Since the writes occurred sequentially, we can
reason that the last object client \(2\) wrote to node \(B\) should be kept.
Some AP-systems, for example Amazon's Dynamo, utilize vector clocks to track
change history and to detect conflicts~\cite{decandia2007dynamo}. Vector clock is
an algorithm to track partial ordering of distributed
events~\cite{fidge1987timestamps}. It is consisted of a timestamp array with an
integer clock value of each process in the network. Each time the process
receives an internal event, the clock is incremented by one. Also when the array
is updated, each element is set to contain the maximum of two corresponding
values, one stored locally and one received from the sender. The value
corresponding the sender is an exception, and it is set to be one greater than
the value received, but only if the local value is not already greater than the
received value.

In practice this means that each object in the system is assigned with a vector
clock that is updated with the write request. Even during network partitions we
can keep track with vector clocks of the casual ordering and not mark sequential
writes as conflicts.

\begin{figure}[h!]
  \centering
    \includegraphics[scale=0.7]{pictures/apexample_clocks.png}
  \caption{Two clients updating the same object during network partition using
  vector clocks}
\label{ap-example-clocks}
\end{figure}

Figure~\ref{ap-example-clocks} shows the same example than above, but this time
the object is assigned with a vector clock. Now after the network partition is
healed, node \(A\) can tell from the vector clock that the change client \(2\)
happened to the object that was originally hold by node \(A\) and no conflict is
needed.


\section{CouchDB}

CouchDB is a distributed database that provides eventual
consistency\cite{anderson2010couchdb}. It is designed around web technologies:
it stores its documents in JSON format and all the communication internally and
externally is done over HTTP\@. CouchDB is modeled around high availability
which categorizes it as AP system.

A CouchDB cluster consists of several identical nodes running CouchDB instances.
A traditional distributed architecture has a single master node that handles all
the writes the system. Leader node then replicates these writes to several
slave nodes that can also respond to read requests by clients. Master/slave
architecture does not scale well to large scale for two reasons. First, since
all the write requests go through a single node, that node need to have enough
processing power to handle all the request. Only way to scale to larger amount
of write requests is to scale vertically which gets very expensive for high-end
hardware. Second, master node acts as a single point of failure in the system.
When master node fails, the distributed system can not process any write
requests until it selects a new master.

Instead of master/slave architecture, each node in CouchDB cluster acts as an
independent instance that can handle reads and writes. This enables high
availability for the cluster: even if several nodes fail, the system is still
able to process requests. However as proven by CAP theorem, the high
availability comes with a cost: the system is no longer able to maintain
consistency between the nodes. This can affect user in certain ways, for example
user can write a document to the database and get \texttt{not\_found} error when
trying to read it moments later. This happens because database has not had time
to replicate the write operation to other node, which user used to read the
document.

Replication in CouchDB is done asynchronously. The replication can happen
periodically or after a replication command from the user depending on the
configuration. Upon replication CouchDB compares the two databases to find out
which documents on the source differ from the target and then sends these
changes as batches until all the changes are transferred.

To find out which documents has changed, databases in CouchDB have a sequence
number which is incremented every time the database is changed. That way CouchDB
is able to tell efficiently what happened between two sequences and can send
only changed data during replication.

The client receives a successful response from the server after a single node
have processed the query. It makes no guarantees that any other node have
received or processed the query because replication takes place later. This
makes both, processing the query and replication straight forward and provides
highly available system. However, because no guarantees about replication is
made, the system can lose all the updates after last replication if the data on
the single machine is corrupted for example due a hardware failure.

It is possible to model replication differently and still have an AP system.
Riak is a distributed database that is designed based on Amazon's Dynamo
architecture\cite{decandia2007dynamo}. Riak stores objects associated with a key
to the database. It uses consistent hashing\cite{karger1997consistent} to map
the keys to the data so that single machine can acts as many virtual nodes in
the hash ring. Riak replication can be configured with multiple factors. Riak
can be configured to replicate the data up to \texttt{N} nodes in the ring, and
clients can define \texttt{R} value upon read request. Riak returns the value
associated with the key when it receives a value from \texttt{R} nodes of
\texttt{N}. For write request the same value is called \texttt{W}. Now when
client sends a write request, it can define to how many nodes Riak needs to
replicate the data until the request is considered successful. To simulate
CouchDB's behaviour, \texttt{R=W=1} could be used, but to achieve stronger
durability a greater value should be used.

% TODO: clustering
%Another problem\todo{if this is another problem, we need to introduce
%first problem (durability)} with large distributed systems is that data can be

\section{Test setup}

Explain the setup that was used for experiments.

\begin{itemize}
  \item Vagrant, Chef
  \item salticid
  \begin{itemize}
    \item tasks
  \end{itemize}
  \item test scripts
\end{itemize}

\section{Experiments with CouchDB}

CouchDB cluster is a homogeneous cluster where every node is allowed to handle
read and write requests from clients. By default, CouchDB does not replicate the
objects written to any other nodes. However it is possible to trigger the
replication either manually or automatically.

The replication logic of CouchDB was tested by first writing numbers from 1 to
2000 to the \texttt{lab1} and then reading triggering the replication. After the
replication was done, the same numbers were read back from other machines. The
same test was ran first without tampering with the network. As expected, in this
case CouchDB replicated all the numbers correctly to the other machines, as can
been seen from listing~\ref{listing-simplecase}.

\begin{lstlisting}[caption={CouchDB replication without network anomalities},label={listing-simplecase}]
vagrant@vagrant-ubuntu-trusty-64:/vagrant$ ruby tests/replication.rb
Writing numbers 0-2000
1/2000 done.
2/2000 done.
3/2000 done.
...
1999/2000 done.
2000/2000 done.
Done, 2000 acknowledged. <ENTER> to continue

Fetching result from lab2
Got results. Got 2000 writes back
Fetching result from lab3
Got results. Got 2000 writes back
Fetching result from lab4
Got results. Got 2000 writes back
Fetching result from lab5
Got results. Got 2000 writes back
\end{lstlisting}

Next the network was partitioned in a way that \texttt{lab1} and \texttt{lab2}
were partitioned from rest of the cluster before starting the test. Because
CouchDB does not require that any other node in the cluster has received the
object before acknowledging the write, the client got successful responses from
all the writes. Then the replication was triggered, but due the network
partition, \texttt{lab1} was only able to replicate the objects to
\texttt{lab2}. Rest of the nodes returned \texttt{\{"error":"timeout"\}}
indicating that the server was not able to reach the other nodes. When trying to
read the numbers back from the machines, only \texttt{lab2} was able to return
them, other nodes timed out. After the network partition was healed and the
replication was triggered and finished again, all the nodes were able to return
the numbers. This is presented in the listing~\ref{listing-partition}

\begin{lstlisting}[caption={CouchDB replication with partitioned
network},label={listing-partition}]

vagrant@vagrant-ubuntu-trusty-64:/vagrant$ ruby tests/replication.rb
Writing numbers 0-2000
1/2000 done.
2/2000 done.
3/2000 done.
...
1999/2000 done.
2000/2000 done.
Done, 2000 acknowledged. <ENTER> to continue

Fetching result from lab2
Got results. Got 2000 writes back
Fetching result from lab3
Timed out.
Fetching result from lab4
Timed out.
Fetching result from lab5
Timed out.

Healing the partition and trying to fetch the results again...

Fetching result from lab2
Got results. Got 2000 writes back
Fetching result from lab3
Got results. Got 2000 writes back
Fetching result from lab4
Got results. Got 2000 writes back
Fetching result from lab5
Got results. Got 2000 writes back
\end{lstlisting}

Same experiments were also performed using automatic replication with following
settings:

\begin{verbatim}
{
  source: "test",
  target: "http://lab_host:5984/test",
  continuous: true,
  connection_timeout: 1000,
  retries_per_request: 1
}
\end{verbatim}

The results were identical to the manual version. Without any anomalies, CouchDB
started to replicate the objects immediately. With the test case where the
network was partitioned, CouchDB started to replicate the objects right after
the partition was healed.

Finally the replication was also tested over a lossy network, that was slow and
dropped packets. CouchDB was able to handle the situations fairly well. In every
case the database was able to return correct results, but all of the operations
were much slower. The lousy network was not problem for CouchDB, because it
communicates over HTTP-protocol, that uses TCP, which has built-in support for
resending and fixing order of packets.

Also CouchDB's ability to detect and resolve conflicts was tested. Because
the database acknowledges the writes without requiring a consensus from other
nodes of the cluster first, it is possible that two clients update the same
resource at the same time in different nodes. When the nodes try to replicate
the object to the other nodes, a conflict may occur if two nodes have a
different version of the same object. However as discussed in AP-systems
section, when implemented properly the database should be able to detect if
updates to the one object happened sequentially and not mark them as conflicted.

First the real conflict was tested by writing an object with same id but with
different payload to \texttt{lab1} and \texttt{lab2}. Then replication was
triggered manually on \texttt{lab1}. Conflicting objects were queried using a
simple view expressed in listing~\ref{listing-conflict-view}. The function maps all the conflicting documents in an array, that can be later queried with CouchDB.

\begin{lstlisting}[caption=A view to query CouchDB
conflicts,label=listing-conflict-view]
function(doc) {
  if(doc._conflicts) {
    emit(doc._conflicts, null);
  }
}
\end{lstlisting}

As expected, after an successful replication, CouchDB detected that the two
different versions for the same object were conflicting and marked them as
conflicted:

\begin{lstlisting}
vagrant@vagrant-ubuntu-trusty-64:/vagrant$ ruby tests/conflicts.rb
Saving id: "1", payload "a" to lab1
Saving id: "1", payload "b" to lab2
Press <ENTER> to query conflicts

[{"id"=>"1", "key"=>["1-0ffc86a53ca35716ebb1f08d9e218819"], "value"=>nil}]
\end{lstlisting}

Secondly CouchDB's ability to detect sequential but distributed updates to the
same object was tested. The test case followed the same structure that was
presented in image~\ref{ap-example}. First client wrote an object to
\texttt{lab1} and it was replicated to \texttt{lab2}. Then it updated the
object on \texttt{lab1} but it was not replicated to the \texttt{lab2}. Finally
another client read the object from \texttt{lab1} and updated it once more, but
this time to \texttt{lab2}. Because all the updates were sequential and not
really conflicting, the database should be able to determine a correct ordering
of events and not mark the two objects conflicting when replicating the objects.
Instead the last write \texttt{lab2} should win and be kept. Instead of
resolving the conflict properly, CouchDB flagged the second update to
\texttt{lab2} as conflicted as shown in listing~\ref{listing-conflict}.

\begin{lstlisting}[caption=CouchDB detecting false positive conflict when updating the same object in different machines,label=listing-conflict]
vagrant@vagrant-ubuntu-trusty-64:/vagrant$ ruby tests/conflicts.rb
Saving id: 1, payload 'a' to lab1
Press <ENTER> to update payload

Updating id: 1 to payload 'b' to lab1
Reading the payload from lab1 and updating it in lab2
409 Conflict (RestClient::Conflict)
\end{lstlisting}

Actually what happened was that CouchDB did not even acknowledge the second
update, because it was read from the another node. This harms the
availability guarantee, because the client is not able to update an object but
in the same node it was originally read. If that node goes unavailable, clients
or even the cluster have no way of trying to route the update request to an
available node.

\section{Conclusion}

Wrap up of the report/research. 

\begin{itemize}
  \item What was done
  \item explain results
  \item future work?
\end{itemize}


\newpage
\addcontentsline{toc}{section}{\refname}  % article

\bibliography{references.bib}
\bibliographystyle{aaltosci_t}
\end{document}
