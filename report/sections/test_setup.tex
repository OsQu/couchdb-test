\section{Experiment setup}

\missingfigure{Figure of the test cluster}

For the experiment, five virtual servers were deployed on the private network
using Vagrant. The servers ran Ubuntu 14.04 and CouchDB 1.5.0 was installed
using Chef.

CouchDB was installed using standard settings and each server communicated
directly with each other.

Two kinds of disturbances were simulated in the test network. First the network
was partitioned in a way, that servers 1 to 3 could not communicate with servers
4 and 5, but were able to communicate directly. Second disturbance was to slow
down the network and to cause it to lose packets.

The network partition was done using \texttt{iptables} to configure the kernel
to drop packets from corresponding servers. For example to disconnect servers 1
and 2 from rest of the cluster, following command was executed:

\begin{verbatim}
  iptables -A INPUT -s <server_host> -j DROP
\end{verbatim}

where \texttt{server\_host} was a host of the server stored in
\texttt{/etc/hosts}.

Network slowness and packet loss was caused using \texttt{tc} utility. It is a
tool used to configure Traffic Control in the Linux kernel and can be used
simulate different network problems. To configure a network interface to slow
transmission of the packets, following command was executed:

\begin{verbatim}
  tc qdisc add dev eth1 root netem delay 100ms 20ms distribution normal
\end{verbatim}

This causes \texttt{eth1} network interface to delay packets by 100ms\(\pm\)20ms
using a normal distribution. In a similar manner, packet loss was caused by
executing a following command:

\begin{verbatim}
  tc qdisc add dev eth1 root netem loss 20% 25%
\end{verbatim}

This causes \texttt{eth1} network interface to lose 20\% of packets, and each
successive probability depends by 25\% of the previous one.

The commands were executed to the servers using
Salticid\footnote{https://github.com/aphyr/salticid}. It is a relatively simple
tool used to run commands on multiple servers over SSH\@. The Salticid roles and
tasks used in the experiment are located on
GitHub\footnote{https://github.com/OsQu/couchdb-test}.
