\section{Introduction}

CouchDB\footnote{http://couchdb.apache.org/} is a distributed database that is
designed around web technologies. Its API relies completely on HTTP protocol and
documents are stored in JSON format. CouchDB also promises to be highly
available, partition tolerant and eventually consistent.

Distributed systems are often categorized to two categories based on their
attributes. The categorization is made using CAP
theorem~\cite{gilbert2002brewer}. CAP theorem states that it is impossible for a
database to simultaneously guarantee \emph{consistency}, \emph{availability}
and \emph{partition tolerance}. These properties are highly desirable for a
distributed system.  Consistency in the CAP theorem means that there exists an
order of operations that is equivalent as if all the operations were completed
by a single instance. In practice this means that the results of operations the
distributed system completes need to be communicated to the every node of the
system. For CAP theorem to consider a distributed system available, it needs to
response to every request received by non-failing node. Partition tolerance
means that the system needs to tolerate partitions in the network i.e.\ the
network can drop arbitrary amount of packets and the system should still
remain functional.

Since it is impossible to build a perfect network, network partitions will occur.
Based on the CAP theorem, a distributed system can not be consistent and
available at the same time, so distributed systems need to choose whether they
keep their data consistent at all times and refuse to response to some of the
queries, or remain available during partition but give up on the data
consistency. Systems that choose to be consistent during network partition are
categorized as CP, and systems that choose to be available are categorized as
AP.\

CP systems can, e.g.\ guarantee consistency by making sure that majority of
non-failing nodes in the cluster has received the operation before sending
client a successful response using a distributed consensus protocol such as
Paxos~\cite{lamport1998part} or Raft~\cite{ongaro2014search}. In case of
network partition the system is split in to two ore more smaller networks. The
partition with majority of nodes can still accept operations and they are
replayed to nodes in the minority partitions when the partition heals. However
nodes in the minority partition can not accept any operations, because they
could be overridden by majority when the network heals and that would violate
consistency.

AP systems guarantee availability by always accepting operations, including
writes, to every non-failing node. This causes the system to be inconsistent
because some previous operations might not have been reached the node
processing a request so it is operating on inconsistent view of the system.
However in case of a network partition, AP system will continue responding to
requests even from the minority partition. When the partition heals, the system
needs a way to solve conflicting operations to the same document that happened
on the different sides of the partition\todo{Mention something about conflict
resolution}. The simplest way is to always choose the most recently timestamped
operation using Last-Write-Wins (LWW). The problem with LWW is that it will
discard the losing writes and that could potentially cause the system to lose
data.  Another approach is to try to merge conflicting operations. However it
can be shown that there exists no universal merge function\todo{citation
needed}. Some AP systems allows a user to write their own merge function which
is applied when the system detects a conflict.  Based on the business
requirements of the user, this might or might be not feasible. It is also
possible to use data types that eliminate the possibility of conflicts.

Letia et al.\ introduced Commutative Replicate Data Types
(CRDT)~\cite{letia2009crdts} which ensures that there are no conflicts between
different versions of the data. They show that if concurrent updates to some
datum commute and all the replicas execute all updates in causal order, then the
replicas will converge. This means that the system does not have to remain
consistent at all times in order to reach consistency eventually. A simple
example of a CRDT is \emph{set} with only addition operation. Since addition
operation commute, it can be applied to all replicas in different order and
still the system would reach the same final set contents. Problem with CRDTs
are that they are not universal. While currently known CRDTs can be quite
complex, they do not offer a solution for every problem thus their usage need
to be considered case by case.
