\section{Introduction}

CouchDB\footnote{http://couchdb.apache.org/} is a distributed database that is
designed around web technologies. Its API relies completely on HTTP protocol and
documents are stored in JSON format. CouchDB also promises to be highly
available, partition tolerant and eventually consistent.

Distributed systems are often categorized to two categories based on their
attributes. The categorization is made using CAP
theorem\cite{gilbert2002brewer}. CAP theorem states that it is impossible for a
web service to simultaneously guarantee consistency, availability and partition
tolerance. These properties are highly desirable for a distributed system.
Consistency in the CAP theorem means that there exists an order of operations
that is equivalent as if all the operations were completed by a single
instance. In practice this means that the results of operations the distributed
system completes need to be communicated to the every node of the system. For
CAP theorem to consider a distributed system available, it needs to response to
every request received by non-failing node. Partition tolerance means that the
system needs to tolerate partitions in the network i.e.\ the network
can drop arbitrary amount of packages and the system should still remain
functional. 

Since perfect network is impossible to build, network partitions will occur.
Based on the CAP theorem, a distributed system can not be consistent and
available at the same time, so distributed systems need to choose whether they
keep their data consistent at all times and refuse to response to some of the
queries, or remain available during partition but give up on the data
consistency.

