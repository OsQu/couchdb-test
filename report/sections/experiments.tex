\section{Experiments with CouchDB}

CouchDB cluster is a homogeneous cluster where every node is allowed to handle
read and write requests from clients. By default, CouchDB does not replicate the
objects written to any other nodes. However it is possible to trigger the
replication either manually or automatically.

The replication logic of CouchDB was tested by first writing numbers from 1 to
2000 to the \texttt{lab1} and then reading triggering the replication. After the
replication was done, the same numbers were read back from other machines. The
same test was ran first without tampering with the network. As expected, in this
case CouchDB replicated all the numbers correctly to the other machines, as can
been seen from listing~\ref{listing-simplecase}.

\begin{lstlisting}[caption={CouchDB replication without network anomalities},label={listing-simplecase}]
vagrant@vagrant-ubuntu-trusty-64:/vagrant$ ruby tests/replication.rb
Writing numbers 0-2000
1/2000 done.
2/2000 done.
3/2000 done.
...
1999/2000 done.
2000/2000 done.
Done, 2000 acknowledged. <ENTER> to continue

Fetching result from lab2
Got results. Got 2000 writes back
Fetching result from lab3
Got results. Got 2000 writes back
Fetching result from lab4
Got results. Got 2000 writes back
Fetching result from lab5
\end{lstlisting}

Next the network was partitioned in a way that \texttt{lab1} and \texttt{lab2}
were partitioned from rest of the cluster before starting the test. Because
CouchDB does not require that any other node in the cluster has received the
object before acknowledging the write, 

- TODO: Get output from this and see what happens when partition is healed, then
test conflict resolution.
