\section{Conclusion}

The paper investigated different characteristics of distributed systems.
Distributed systems can be divided into two categories based on the
CAP-theorem. CAP-theorem states that it is not possible for a distributed system
to stay available and consistent at the same time. Availability means that the
system is able to respond to every request either success or failure, and
consistent means that all nodes of the system see the same data at the same
time.

CP-systems aim to keep their data consistent at all times, and in the events of
network partitions they give up availability. AP-systems on the other hand aim
to remain available and might end up in a inconsistent state during a network
partition. Neither of the approaches are ideal and contain their cave-eats that
were discussed in this paper.

This paper also studied an existing AP-system, a database called CouchDB. The
database is a distributed object storage that heavily relies on web technologies
such as HTTP and JSON. CouchDB has an extensive documentation that covered areas
from the basic usage to scaling the server cluster using replication mechanics
and clustering. However in practice, the current state of CouchDB is worse than
was documented. The database was able to replicate documents between servers as
promised, but marked some updates to the objects incorrectly as conflicted, even
though with proper implementation it could have been avoided. Also, despite what
was stated in the documentation, no clustering solution was available any more.

Even though the replication in CouchDB worked decently, lack of clustering and
relatively simple conflict detection algorithm make it unsuitable to be used in
any real production system.
