\section{CouchDB}

CouchDB is a distributed database that provides eventual
consistency\cite{anderson2010couchdb}. It is designed around web technologies:
it stores its documents in JSON format and all the communication internally and
externally is done over HTTP\@. CouchDB is modeled around high availability
which categorizes it as AP system.

A CouchDB cluster consists of several identical nodes running CouchDB instances.
A traditional distributed architecture has a single master node that handles all
the writes the system. Leader node then replicates these writes to several
slave nodes that can also respond to read requests by clients. Master/slave
architecture does not scale well to large scale for two reasons. First, since
all the write requests go through a single node, that node need to have enough
processing power to handle all the request. Only way to scale to larger amount
of write requests is to scale vertically which gets very expensive for high-end
hardware. Second, master node acts as a single point of failure in the system.
When master node fails, the distributed system can not process any write
requests until it selects a new master.

Instead of master/slave architecture, each node in CouchDB cluster acts as an
independent instance that can handle reads and writes. This enables high
availability for the cluster: even if several nodes fail, the system is still
able to process requests. However as proven by CAP theorem, the high
availability comes with a cost: the system is no longer able to maintain
consistency between the nodes. This can affect user in certain ways, for example
user can write a document to the database and get \texttt{not\_found} error when
trying to read it moments later. This happens because database has not had time
to replicate the write operation to other node, which user used to read the
document.

Replication in CouchDB is done asynchronously. The replication can happen
periodically or after a replication command from the user depending on the
configuration. Upon replication CouchDB compares the two databases to find out
which documents on the source differ from the target and then sends these
changes as batches until all the changes are transferred.

To find out which documents has changed, databases in CouchDB have a sequence
number which is incremented every time the database is changed. That way CouchDB
is able to tell efficiently what happened between two sequences and can send
only changed data during replication.

The client receives a successful response from the server after a single node
have processed the query. It makes no guarantees that any other node have
received or processed the query because replication takes place later. This
makes both, processing the query and replication straight forward and provides
highly available system. However, because no guarantees about replication is
made, the system can lose all the updates after last replication if the data on
the single machine is corrupted for example due a hardware failure.

It is possible to model replication differently and still have an AP system.
Riak is a distributed database that is designed based on Amazon's Dynamo
architecture\cite{decandia2007dynamo}. Riak stores objects associated with a key
to the database. It uses consistent hashing\cite{karger1997consistent} to map
the keys to the data so that single machine can acts as many virtual nodes in
the hash ring. Riak replication can be configured with multiple factors. Riak
can be configured to replicate the data up to \texttt{N} nodes in the ring, and
clients can define value \texttt{R} upon read request. Riak returns the value
associated with the key when it receives a response from \texttt{R} nodes of
\texttt{N}. For write request the same value is called \texttt{W}. When the
client sends a write request, it can define to how many nodes Riak needs to
replicate the data until the request is considered successful. To simulate
CouchDB's behaviour, \texttt{R=W=1} could be used, but to achieve stronger
durability a greater value should be used.

When CouchDB replicates updates from one server to another, it is possible that
the same document was updated in both servers. The document can not be just
transferred because that would override the change that was made on the target
server. Instead, CouchDB marks the document as conflicted and deterministically
stores both versions in a way that they are same in both servers. Then the user
is required to resolve the conflict by either choosing one of the versions or
trying to merge them.

CouchDB detects conflicts between versions by comparing their \emph{revision
ids}. The database assigns revision id to each document it stores, and it is
increment by one each time the document updated. Because multiple servers can
update the same revision independently, the revision id also contains a MD5 hash
of the data stored in the document. If the documents have different revision ids
they are conflicting.\todo[inline]{Now there's plenty of weird stuff here: ID is
basically \texttt{<rev\_num>-MD5(document)}. Is the algorithm able to notice
consecutive edits that do not conflict and not mark them, what about emerging
branches with different revision numbers and conflicting data? Riak handles
these with dotted version vectors, Dynamo with vector clocks.}

% TODO: Clustering
% Another problem\todo{if this is another problem, we need to introduce
% first problem (durability)} with large distributed systems is that 
